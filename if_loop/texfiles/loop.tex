\subsection{Loops}
\begin{frame}[fragile]
    \frametitle{Loops: definition}
    Loops should be used when a set of actions should be repeated a certain number of times (for instance on each element of a list).\\
    \vspace{1em}

    There is mainly two ways to perform a loop: by using a \verb+for+ statement, or by using a \verb+while+ statement.

    \vspace{1em}
    In Python, the general structure of a loop is:
    \begin{lstlisting}[basicstyle=\ttfamily\scriptsize]
for v in iterable:
    action

while(condition):
    action
    \end{lstlisting}

    \vspace{1em}
    To play with loops, open the \verb+loops.py+ file on Spyder

    \begin{block}{Note}
        You can always replace a \verb+for+ loop by a \verb+while+ loop, and conversely. 
    \end{block}
\end{frame}

%
%\begin{frame}[fragile]
%\frametitle{Loops}
%\begin{lstlisting}
%x = range(1, 11) # array from 1 to 10
%
%for p in range(0, len(x)): # note the colon!
%    print x[p] # p: index of the element
%    # inside a loop, the indent must change!
%# no end statement, just left indent
%
%for temp in x:
%    print temp # temp: element itself
%
%cpt = 0 # initialisation of counter
%while cpt < len(x):
%    print x[cpt]
%    cpt += 1 # iterate counter
%\end{lstlisting}
%\end{frame}
%
%\begin{frame}[fragile]
%\frametitle{Loops}
%\begin{lstlisting}
%# loop over multiple lists
%x = range(1, 11); y = range(11, 21); z = range(21, 31)
%zip(x,y,z) # [(1,11,21), (2,12,22) ... ]: list of x,y,z tuples
%for xtemp, ytemp, ztemp in zip(x, y, z):
%    print xtemp, ytemp, ztemp # 1 11 21; 2 12 22; ...
%
%# Loop on 2D arrays
%x = np.empty((50, 50))
%for i in xrange(0, x.shape[0]):
%    for j in xrange(0, x.shape[1]):
%        temp = x[i, j]
%# xrange instead of range
%# allows to avoid the creation
%# of a list in memory (lazy eval)
%\end{lstlisting}
%\end{frame}
%
%\begin{frame}[fragile]
%\frametitle{Special loops (List Comprehensions)}
%
%Very efficient loops, which output a list:
%
%\begin{lstlisting}
%combs1 = [(x, y) for x in [1,2,3] for y in [3,1,4] if x != y]
%# [(1, 3), (1, 4), (2, 3), (2, 1), (2, 4), (3, 1), (3, 4)]
%
%# equivalent to:
%combs2 = []
%for x in [1, 2, 3]:
%    for y in [3, 1, 4]:
%        if x != y:
%            combs2.append((x, y))
%
%x = range(1, 10)
%y = [3, 5, 6]
%# extract x values that are not in y
%z = [temp for temp in x if temp not in y]
%# [1, 2, 4, 7, 8, 9]
%\end{lstlisting}
%\end{frame}
%
%\begin{frame}[fragile]
%\frametitle{Break and continue statements}
%\begin{lstlisting}
%for p in xrange(0, 10):
%    
%    print p
%
%    if(p > 3):
%        # if p > 3, the loop is exited.
%        # hence, loops stops for p=4, and not 10
%        break
%
%for p in xrange(0, 10):
%    
%    print p
%   
%    if(p > 3):
%        # if p > 3, the statements below are not
%        # excuted.
%        # hence, 'below' is printed for p in [0, 3]
%        # but loop is performed until p=9
%        continue
%    
%    print 'below'
%\end{lstlisting}
%\end{frame}
